%----------------------------------------------------------------------------
\section{Fizika 1}
{\footnotesize Fizikai fogalmak, mennyiségek. Impulzus, impulzusmomentum. Newton törvényei. Munkatétel. Az I. és II. főtétel. A kinetikus gázmodell.}
%----------------------------------------------------------------------------
\subsection{Fizikai fogalmak, mennyiségek.}
\paragraph{Alapfogalmak}
\begin{description}[nosep]
	\item[Mérőszám] Megmutatja, hogy a mértékegységet hányszor lehet a mérendő mennyiségbe belefoglalni.
	\item[Mértékegység] Egy mérőszám típusát és nagyságrendjét meghatározó jelző.
	\item[Fizikai mennyiség] Adott fizikai jellemzőt leíró mérőszám és annak mértékegysége. Valamely jelenség, folyamat minőségileg megkülönböztethető, és mennyiségileg meghatározható tulajdonsága.
	\begin{enumdescript}[nosep]
		\item[Skalármennyiség] Olyan mennyiség, melynek nincs iránya, tehát teljes mértékben leír a nagysága (pl. hőmérséklet, tömeg, térfogat).
		\item[Vektormennyiség] Olyan mennyiség, melynek nagyságán kívül iránya is van (pl. erő, sebesség, térerősség).
	\end{enumdescript}
\end{description}

\paragraph{Erő}
\emph{Olyan hatás, ami egy tömeggel rendelkező testet gyorsulásra késztet.} Az \emph{eredő erő} a testre ható \emph{összes erő vektoriális összege}. $1N$ erő olyan hatás, amely egy $1kg$$  $ tömegű testet $1\frac{m}{s^2}$ mértékű gyorsulásra késztet. \underline{Vektormennyiség}. Iránya megegyezik a gyorsulás irányával. Néhány alapvető erő: elektromágneses erő, gravitációs erő, nukleáris erő.\\
Jele: $F$ (force) \quad $\vec{F} = m \cdot \vec{a}$\\
Mértékegysége: $N$ (Newton) \quad $ N = \nicefrac{kg\cdot m}{s^2} $

\paragraph{Tömeg}
\emph{A fizikai testek tulajdonsága, amely a tehetetlenségüket méri.} A súlytól eltérően a tömeg mindig ugyanaz marad, akárhová kerül is a hordozója. \underline{Skalármennyiség}.\\
Jele: $m$ (mass)\\
Mértékegysége: $kg$ (kilogramm)

Szigorúan véve három különböző dolgot neveznek tömegnek:
\begin{description}
	\item[Tehetetlen tömeg] a test tehetetlenségének mértéke: a rá ható erő mozgásállapot változtató hatásával szembeni ellenállás. A kis tehetetlen tömegű test sokkal gyorsabban változtatja mozgásállapotát, mint a nagy tehetetlen tömegű.
	\item[Passzív gravitáló tömeg] a test és a gravitációs tér kölcsönhatásának mértéke. Azonos gravitációs térben a kisebb passzív gravitáló tömegű testre kisebb erő hat, mint a nagyobbra.
	\item[Aktív gravitáló tömeg] a test által létrehozott gravitációs tér erősségének a mértéke. Például a Hold gyengébb gravitációs teret hoz létre, mint a Föld, mert a Holdnak kisebb az aktív gravitáló tömege.
\end{description}

\paragraph{Súly}
Az az \underline{erő}, amellyel a test az \textbf{alátámasztást nyomja} vagy a \textbf{felfüggesztést húzza}, tehát \emph{a test környezetére gyakorolt erőinek az eredője}. Azonos tömegű testek különböző erősségű gravitációs terekben különböző súlyúak, mert a testre ható gravitációs erőt a test közvetíti a környezetének.\\
Jele: $G$ \quad $\vec{G} = m \cdot \vec{g}$\\
Mértékegysége: $N$ (Newton)

\paragraph{Távolság}
Két pont közötti távolság az a legrövidebb út, melyen eljutunk egyik pontból a másikba, tehát a helyváltozás mértéke. A legrövidebb út általában egyenes mentén van. \underline{Skalármennyiség}.\\
Jele: $d$ (distance) vagy $l$ (length)\\
Mértékegysége: $m$ (méter)

\paragraph{Pálya}
Azt a vonalat, amin a test mozog, pályának nevezzük.

\paragraph{Út}
A mozgó test által befutott pályaszakasz hossza a megtett út.

\paragraph{Elmozdulás}
A kezdőpontból a végpontba mutató \underline{vektort} elmozdulásnak nevezzük.
Az út hossza nem lehet kisebb az elmozdulás nagyságánál, hiszen két pont között az egyenes szakasznak a legkisebb a hossza.

\paragraph{Idő}
A folyamatokban bekövetkező \emph{események sorrendiségének kifejezésére való skalármennyiség}. Az idő SI-alapegysége az SI-másodperc(?). Az ebből származó nagyobb időegységek, mint perc, óra és nap, nem SI-egységek, mert nem tízes számrendszerűek, és szükség van időnként szökőmásodpercre.\\
Jele: $t$ (time)\\
Mértékegysége: $s$ (szekundum vagy másodperc)

\paragraph{Nyomás}
Az egységnyi felületre ható erőhatást adja meg.\\
Jele: $p$ (pressure) \quad $p = \nicefrac{F}{A}$\\
Mértékegysége: $Pa$ (Pascal) \quad $Pa = \nicefrac{N}{m^2}$

\paragraph{Munka}
Amikor egy \emph{testre kifejtett erő hatására a test elmozdul, mechanikai munkavégzés történik}, ami \underline{arányos} a kifejtett \underline{erő nagyságával} és a \underline{megtett úttal}. A munka az erő és az elmozdulás \underline{skaláris szorzata}. Egy joule munkát végez az egy newton nagyságú erő a vele egyirányú egy méter hosszúságú elmozdulás közben. Ugyancsak egy joule az egy watt teljesítménnyel egy másodpercig végzett munka.\\
Jele: $W$ (work) \quad $W=F\cdot s \cdot \cos\alpha$\\
Mértékegysége: $J$ (Joule)	\quad $J = N \cdot m = kg \cdot \frac{m^2}{s^2}$

\paragraph{Teljesítmény}
A munkavégzés vagy energiaátvitel sebessége, más szóval az \emph{egységnyi idő alatt végzett munka}.\\
Jele: $P$ (power) \quad $P = \nicefrac{W}{t}$\\
Mértékegysége: $W$ (Watt) \quad $W = \nicefrac{J}{s}$

\paragraph{Sebesség}
Egy pontszerű test kitüntetett ponthoz viszonyított mozgásának jellemzésére szolgáló fizikai mennyiség. \textbf{Az út idő szerinti deriváltja}, tehát az \emph{időegység alatt bekövetkezett helyváltozás mértéke}. \underline{Vektormennyiség}.\\
Jele: $v$ (velocity) \quad $v = \nicefrac{s}{t}$\\
Mértékegysége: $\nicefrac{m}{s}$

\paragraph{Gyorsulás}
\textbf{A sebességvektor idő szerinti deriváltja}, tehát az \emph{időegység alatt bekövetkezett sebességváltozás mértéke}. \underline{Vektormennyiség}.\\
Jele: $a$ (acceleration) 
$$\overline{a}=\frac{\Delta v}{\Delta t} = \frac{v-u}{t}\quad \text{ahol v a végsebesség, u a kezdeti sebesség}$$
$$a=\lim\limits_{\Delta t \to 0} \frac{\Delta v}{\Delta t} = \frac{dv}{dt}$$
Mértékegysége: $\nicefrac{m}{s^2}$

\subsection{Impulzus, impulzusmomentum.}
\paragraph{Lendület (impulzus)}
Egy test mozgását leíró vektormennyiség. Nagysága arányos a tömeggel és a sebességgel. Lendületmegmaradás törvénye: zárt rendszer összes lendülete állandó. Nyugalomban lévő testnek nincs lendülete, a lendületet csakis külső erő változtathatja meg.\\
Jele: $I$ (impulzus)	$$\vec{I}=m\cdot \vec{v}$$
Mértékegysége: $kg\cdot \nicefrac{m}{s} = N \cdot s$

\paragraph{Perdület (impulzusmomentum)} Forgómozgásban lévő test lendülete által létrehozott nyomatékot jellemző vektormennyiség. Az erőkar és a lendület szorzata. Zárt rendszerben a lendületmegmaradás következtében a perdület állandó. Rögzített tengely körüli forgásnál a perdületet a tehetetlenségi nyomaték és a szögsebesség szorzatából számíthatjuk ki.
Jele: $L$	$$L=\Theta \cdot \omega$$
Mértékegysége: $kg\cdot \nicefrac{m^2}{s} = N \cdot m \cdot s$

\subsection{Newton törvényei.}
\begin{description}
	\item[I. Tehetetlenség törvénye] Minden inerciarendszerben vizsgált test nyugalomban marad vagy egyenes vonalú egyenletes mozgást végez mindaddig, míg ezt az állapotot egy másik test vagy erő hatása meg nem változtatja egy kölcsönhatás során.

	\item[II. Dinamika alaptörvénye] Egy pontszerű test gyorsulása azonos irányú a rá ható erővel, nagysága egyenesen arányos az erő nagyságával, és fordítottan arányos a test tömegével. $\vec{F}=m \cdot \vec{a}$

	\item[III. Hatás-ellenhatás törvénye] Két test kölcsönhatása során mindkét testre azonos nagyságú, azonos hatásvonalú és egymással ellentétes irányú erő hat.

	\item[IV. Szuperpozíció elve] Ha egy testre egy időpillanatban több erő hat, akkor ezek együttes hatása megegyezik a vektori eredőjük (vektoriális összegük) hatásának vonalával.
\end{description}

\subsection{Munkatétel.}
\begin{theorem}
	Egy test mozgási energiájának változásának mértéke megegyezik a testre ható összes erő munka előjeles összegével.
	$$\Delta E_m = \sum_{i=1}^{n}W_i$$
\end{theorem}

\subsection{A termodinamika I. és II. főtétel.}
\begin{theorem}[I. főtétel]
	Egy zárt rendszer belső energiájának változása egyenlő a rendszerrel közölt hő és a rendszeren végzett munka összegével.
	$$ \Delta U = Q+W $$
\end{theorem}
A testek belső energiájának megváltozása egyenlő a testtel közölt hőmennyiség és a testen végzett munka előjeles összegével. Ez az energiamegmaradás törvénye, mert azt mondja ki, hogy külső beavatkozás nélkül nincs energiaváltozás.

\begin{theorem}[II. főtétel]
	Termikus kölcsönhatással járó természetes folyamatoknál csak a nagyobb hőmérsékletű test képes a hőátadásra. Tehát egy elszigetelt rendszer állapota időben termikus egyensúly felé halad.
\end{theorem}

\subsection{A kinetikus gázmodell.}
A gázok tömeggel rendelkező részecskékből állnak, melyek energiaveszteség nélkül ütköznek egymással és a környezetükben lévő testekkel. Mozgási energiájuk csak a rendszer hőmérsékletétől függ. Állandóan mozgásban vannak, és az ütközések között egyenes vonalú egyenletes mozgást végeznek. Össztérfogatuk mindig jóval kisebb, mint a gázt tartalmazó tároló térfogata.
Alapegyenlet: $$p \cdot V = \frac{2}{3}N \cdot \frac{1}{2}mv^2$$ ahol p az ütközésekkel keltett nyomás, V a térfogat, N a molekulaszám, az $\nicefrac{1}{2}mv^2$ pedig egy molekula átlagos mozgási energiája.

A kinetikus gázmodell segítségével értelmezhetők a gázok olyan makroszkopikus tulajdonságai, mint a nyomás, a hőmérséklet, vagy a térfogat. Newton úgy gondolta, hogy a gáz nyomását a molekulái között fellépő állandó taszítás okozza, a kinetikus gázelmélet szerint viszont a nyomás a különböző sebességgel mozgó gázmolekulák ütközéseiből származik. A kinetikus gázmodell a természetben nem létező ideális gázt feltételez.
A kinetikus gázmodell szerint:
\begin{enumerate}
	\item a gáz részecskékből áll, amelyeknek tömege és súlya van, viszont össztérfogatuk elhanyagolható a gázt tartalmazó edény térfogatához képest
	\item a részecskék állandó mozgásban vannak
	\item a részecskék tökéletes gömb alakúak és rugalmas természetűek
	\item a részecskék mozgási energiája csak a rendszer hőmérsékletétől függ
	\item a részecskék egymással és az edény falával energiaveszteség nélkül ütköznek
	\item a részecskék közötti erőhatások elhanyagolhatóak, ezért két ütközés között egyenes vonalú egyenletes mozgást végeznek
\end{enumerate}
A kinetikus gázelmélet alapegyenlete:
$$ pV = \frac{2}{3}N \frac{m_0v^2}{2}$$
$p$ – nyomás
$V$ – térfogat
$N$ – molekulák száma
$m_0$ – egy molekula tömege
$\frac{m_0v^2}{2}$– egy molekula átlagos mozgási energiája