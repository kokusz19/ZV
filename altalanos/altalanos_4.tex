%----------------------------------------------------------------------------
\section{Magas Sintű programozási nyelvek 2}
%----------------------------------------------------------------------------
\subsection{Speciális programnyelvi eszközök.}
Nem érdemel külön fejezetet

\subsection{Az objektumorientált programozás eszközei és jelentősége.}
Az objektumorientált (OO) paradigma középpontjában a programozási nyelvek absztrakciós szintjének növelése
áll. Ezáltal egyszerűbbé, könnyebbé válik a modellezés, a valós világ jobban leírható, a valós problémák
hatékonyabban oldhatók meg. Az OO szemlélet szerint az adatmodell és a funkcionális modell egymástól
elválaszthatatlan, külön nem kezelhető. A valós világot egyetlen modellel kell leírni és ebben kell kezelni a
statikus (adat) és a dinamikus (viselkedési) jellemzőket. Ez az egységbezárás elve.

\subsection{Funkcionális és logikai programozás.}
