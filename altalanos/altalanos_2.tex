%-------------------------------------------------------------------------------
\section{Operációs rendszerek}
%-------------------------------------------------------------------------------
\subsection{Operációs rendszerek fogalma, felépítése, osztályozásuk.}
\paragraph{Operációs rendszerek fogalma}
Egy program, amely közvetítő szerepet játszik a számítógép felhasználója
és a számítógéphardver között.
Az operációs rendszer feladata, hogy a felhasználónak egy olyan egyenértékű kiterjesztett
vagy virtuális gépet nyújtson, amelyiket egyszerűbb programozni, mint a mögöttes hardvert
\paragraph{Operációs rendszerek felépítése}
Az operációs rendszerek alapvetően három részre bonthatók:
	\begin{itemize}[nosep]
	\item a felhasználói felület (a shell, amely lehet egy grafikus felület, vagy egy szöveges)
	\item alacsony szintű segédprogramok
	\item kernel (mag), amely közvetlenül a hardverrel áll kapcsolatban.
	\end{itemize}
\paragraph{Operációs rendszerek osztályozása}
	\begin{enumerate}[nosep]
	\item Az operációs rendszer alatti hardver "mérete" szerint:
		\begin{itemize}[nosep]
		\item mikroszámítógépek operációs rendszerei
		\item kisszámítógépek, esetleg munkaállomások operációs rendszerei
		\item nagygépek (Main Frame Computers, Super Computers) operációs rendszerei
		\end{itemize}
	\item A kapcsolattartás típusa szerint:
		\begin{itemize}[nosep]
		\item kötegelt feldolgozású operációs rendszerek vezérlőkártyás kapcsolattartással
		\item interaktív operációs rendszerek.
		\end{itemize}
	\item cél szerint: általános felhasználású vagy céloperációs rendszer
	\item a processzkezelés: single-tasking, multi-tasking
	\item a felhasználók száma szerint: single, multi
	\item CPU-idő kiosztása szerint: szekvenciális, megszakítás vezérelt, event-polling, time-sharing
	\item a memóriakezelés megoldása szerint: valós és virtuális címzésű
	\end{enumerate}

\subsection{Az operációs rendszerek jellemzése (komponensei és funkciói).}
\paragraph{Operációs rendszerek komponensei:}
\begin{description}[nosep]
	\item[Eszközkezelők (Device Driver)] Felhasználók elől el fedik a perifériák különbségeit, egységes kezelői felületet kell biztosítani.
	\item[Megszakítás kezelés (Interrupt Handling)] Alkalmas perifériák felől érkező kiszolgálási igények fogadására, megfelelő ellátására.
	\item[Rendszerhívás, válasz (System Call, Reply)] az operációs rendszer magjának ki kell szolgálnia a felhasználói alkalmazások (programok) erőforrások iránti igényeit úgy, hogy azok lehetőleg észre se vegyék azt, hogy nem közvetlenül használják a perifériákat$\leftarrow$ programok által kiadott rendszerhívások, melyekre rendszermag válaszokat küldhet.
	\item[Erőforrás kezelés (Resource Management)] Az egyes eszközök közös használatából származó konfliktusokat meg kell előznie, vagy bekövetkezésük esetén fel kell oldania.
	\item[Processzor ütemezés (CPU Scheduling)] Az operációs rendszerek ütemező funkciójának a várakozó munkák között valamilyen stratégia alapján el kell osztani a processzor idejét, illetve vezérelnie kell a munkák közötti átkapcsolási folyamatot.
	\item[Memóriakezelés (Memory Management)] Gazdálkodnia kell a memóriával, fel kell osztania azt a munkák között úgy, hogy azok egymást se zavarhassák, és az operációs renszerben se tegyenek kárt.
	\item[Állomány- és lemezkezelés (File and Disk Management)] Rendet kell tartania a hosszabb távra megőrzendő állományok között.
	\item[Felhasználói felület (User Interface)] A parancsnyelveket feldolgozó monito utódja, 	fejlettebb változata, melynek segítségével a felhasználó közölni tudja a rendszermaggal kívánságait, illetve annk állapotáról információt szerezhet.
\end{description}

\paragraph{Operációs rendszerek funkciói:}
\begin{description}[nosep]
\item[Folyamatkezelés]
A folyamat egy végrehajtás alatt álló program. Hogy feladatát ellássa erőforrásokra van szüksége (processzor idő, memória, állományok I/O berendezések).
Az operációs rendszer feladata:
\begin{itemize}[nosep]
	\item Folyamatok létrehozása és törlése
	\item Folyamatok felfüggesztése és újraindítása
	\item Eszközök biztosítása a folyamatok kommunikációjához és szinkronizációjához.
\end{itemize}
\item[Memória (főtár) kezelés]
Bájtokból álló tömbnek tekinthető, amelyet a CPU és az I/O közösen használ. Tartalma törlődik rendszerkikapcsoláskor és rendszerhibáknál.
Az operációs rendszer feladata:
\begin{itemize}[nosep]
	\item Nyilvántartani, hogy az operatív memória melyik részét ki (mi) használja.
	\item Eldönteni melyik folyamatot kell betölteni, ha memória felszabadul.
	\item Szükség szerint allokálni és felszabadítani a memória területeket a szükségleteknek megfelelően.
\end{itemize}
\item[Másodlagos tárkezelés]
Nem törlődik, és elég nagy hogy minden programot tároljon. A merevlemez a legelterjedtebb
formája. Az operációs rendszer feladata:
\begin{itemize}[nosep]
	\item Szabadhely kezelés.
	\item Tárhozzárendelés.
	\item Lemez elosztás.
\end{itemize}
\item[I/O rendszerkezelés]
\begin{itemize}[nosep]
	\item Puffer rendszer.
	\item Általános készülék meghajtó (device driver) interface.
	\item Speciális készülék meghajtó programok.
\end{itemize}
\item[Fájlkezelés]
Egy fájl kapcsolódó információk együttese, amelyet a létrehozója definiál. Általában program és adatfájlokról beszélünk.
Az operációs rendszer feladata:
\begin{itemize}[nosep]
	\item Fájlok és könyvtárak létrehozás és törlése.
	\item Fájlokkal és könyvtárakkal történő alapmanipuláció.
	\item Fájlok leképezése a másodlagos tárra, valamilyen nem törlődő, stabil adathordozóra.
\end{itemize}
\item[Védelmi rendszer]
Olyan mechanizmus, mely az erőforrásokhoz való hozzá férést felügyeli. Az operációs rendszer feladata:
\begin{itemize}[nosep]
	\item Különbséget tenni jogos (authorizált) és jogtalan használat között.
	\item Specifikálni az alkalmazandó kontrolt.
	\item Korlátozó eszközöket szolgáltatni.
\end{itemize}
\item[Hálózat elérés támogatása]
Az elosztott rendszer processzorok adat és vezérlő vonallal összekapcsolt együttese, ahol a memória és az óra nem közös. Adat- és vezérlővonal segítségével történik a kommunikáció. Az elosztott rendszer a felhasználóknak különböző osztott erőforrások elérését teszi lehetővé, mely lehetővé teszi:
\begin{itemize}[nosep]
	\item a számítások felgyorsítását,
	\item a jobb adatelérhetőséget,
	\item a nagyobb megbízhatóságot.
\end{itemize}
\item[Parancs interpreter alrendszer]
Az operációs rendszernek sok parancsot vezérlő utasítás formájában lehet megadni. Vezérlő utasítások minden területhez tartoznak (folyamatok, I/O kezelés...). Az operációs rendszernek azt a programját, amelyik a vezérlő utasítást beolvassa és interpretálja a rendszertől függően más és más módon nevezhetik:
\begin{itemize}[nosep]
	\item Vezérlő kártya interpreter.
	\item Parancs sor interpreter (command line).
	\item Héj (burok, shell)
\end{itemize}
\end{description}

\subsection{A rendszeradminisztráció, fejlesztői és alkalmazói támogatás eszközei.}
\paragraph{Rendszeradminisztráció}
Magának az operációs rendszernek a működtetésével kapcsolatos funkciók. Ezek közvetlenül semmire sem használhatók, csak a hardverlehetőségek kibővítését célozzák, illetve a hardver kezelését teszik kényelmesebbé. A rendszeradminisztráción belül a következő \emph{összetett funkciókat} jelölhetjük ki:
\begin{enumdescript}[nosep]
	\item[processzorütemezés:] a CPU-idő szétosztása a rendszer- és a felhasználói feladatok
	(taszkok, folyamatok) között;
	\item[megszakításkezelés:] a hardver-szoftver megszakításkérések elemzése, állapotmentés,
	a kezelőprogram hívása;
	\item[szinkronizálás:] az események és az erőforrásigények várakozási sorokba állítása;
	\item[folyamatvezérlés:] a programok indítása és a programok közötti kapcsolatok
	szervezése;
	\item[tárkezelés:] a főtár, -- mint kiemelten kezelt erőforrás, -- elosztása;
	\item[perifériakezelés:] a bemeneti/kimeneti (B/K ill. I/O) igények sorba állítása és
	kielégítése;
	\item[adatkezelés:] az adatállományokon végzett műveletek segítése (létrehozás, nyitás,
	zárás, írás, olvasás stb.);
	\item[működés-nyilvántartás:] a hardver hibastatisztika vezetése és a számlaadatok
	feljegyzése;
	\item[operátori interfész:] a kapcsolattartás az üzemeltetővel.
\end{enumdescript}
A konkrét operációs rendszerek a funkciókat másképpen oszthatják fel. Így például az IBM OS operációs rendszerek változataiban négy fő funkciót szoktak megkülönböztetni:
\begin{enumerate}[nosep]
	\item a munkakezelést,
	\item a taszkkezelést,
	\item az adatkezelést és
	\item a rendszerstatisztikát.
\end{enumerate}
A rendszeradminisztrációs funkciókat a \textbf{rendszermag} valósítja meg, amelynek a szolgáltatásait a már említett rendszerhívásokkal érhetjük el.

\paragraph{Programfejlesztési támogatás} fő funkciói:
\begin{enumdescript}[nosep]
	\item[rendszerhívások:] a programokból alacsony szintű operációsrendszeri funkciók
	aktivizálására,
	\item[szövegszerkesztők:] a programok és dokumentációk írására,
	\item[programnyelvi eszközök:] fordítóprogramok és interpreterek (értelmezők) a nyelvek
	fordítására vagy értelmezésére,
	\item[szerkesztő- és betöltő-programok:] a programmodulok összefűzésére illetve tárba
	töltésére (végcímzés),
	\item[programkönyvtári funkciók:] a különböző programkönyvtárak használatára,
	\item[nyomkövetési rendszer:] a programok belövésére.
\end{enumdescript}

\paragraph{Alkalmazói támogatás}
Az alkalmazói támogatás funkciói a számítógépes rendszer több szintjén valósulnak meg, és az alábbi fő funkciókra bonthatók:
\begin{enumdescript}[nosep]
	\item[operátori parancsnyelvi rendszer:] a számítógép géptermi üzemvitelének
	támogatására;
	\item[munkavezérlő parancsnyelvi rendszer:] a számítógép alkalmazói szintű
	igénybevételének megfogalmazására;
	\item[rendszerszolgáltatások:] az operációs rendszer magjával közvetlenül meg nem oldható
	rendszerfeladatokra;
	\item[segéd-programkészlet:] rutinfeladatok megoldására;
	\item[alkalmazói programkészlet:] az alkalmazásfüggő feladatok megoldására
\end{enumdescript}
