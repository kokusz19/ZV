\documentclass[magyar,\PAPER,\OPTIONS]{article}
\usepackage{imakeidx}    %Tárgymutatóhoz (index)
\usepackage[utf8]{inputenc}
\usepackage[T1]{fontenc}
\usepackage{mathtools}
\usepackage{amsthm}
\usepackage{enumitem}
\usepackage[magyar]{babel}
\usepackage[sharp]{easylist}
\usepackage[
top=2cm,
bottom=2cm,
left=2cm,
right=1cm
]{geometry}
\usepackage[]{hyperref}
%\usepackage{showframe}
%\usepackage{titlesec}

%\titleformat{\section}[display]{\bfseries}{\thesection. tétel:}{0pt}{}[]
\let\OldEasylist\easylist
\let\OldEndEasylist\endeasylist
\renewenvironment{easylist}{
	\OldEasylist
	\ListProperties(Numbers2=l,FinalMark2={)},Hide2=1,Progressive*=3ex, Start1=1)
}{
	\OldEndEasylist
}
\newcounter{descriptcount}
\newcounter{prevdescriptcount}
\newlist{enumdescript}{description}{2}
\setlist[enumdescript,1]{
	before={\setcounter{descriptcount}{0}
		\renewcommand*\thedescriptcount{\arabic{descriptcount}}}
	,font=\bfseries\stepcounter{descriptcount}\thedescriptcount.~
}
\setlist[enumdescript,2]{
	before={\setcounter{prevdescriptcount}{\value{descriptcount}}
		\setcounter{descriptcount}{0}
		\renewcommand*\thedescriptcount{\alph{descriptcount}}}
	,font=\bfseries\stepcounter{descriptcount}\thedescriptcount.~
	,after={\setcounter{descriptcount}{\value{prevdescriptcount}}}
}

\newtheorem*{definition}{Definíció}
\newtheorem*{theorem}{Tétel}
\newtheorem*{note}{Megjegyzés}

\makeindex

\title{Záróvizsga tételsor mérnökinformatikus hallgatóknak\\
	{\large A Debreceni Egyetem mérnökinformatikus alapszakához}}
\author{Palkovics Dénes\thanks{Az egyes tételek kidolgozásai nem tőlem származnak. Az esetenként előforduló hibákért és pongyola fogalmazásért felelősséget nem vállalok.} \TARSSZERZO}
\date{2019}
