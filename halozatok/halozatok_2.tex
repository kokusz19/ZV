%----------------------------------------------------------------------------
\section{Hálózatok hatékonyságanalízise}
{\footnotesize Markov-láncok, születési-kihalási folyamatok. A legalapvetőbb sorbanállási rendszerek vizsgálata. A rendszerjellemzők meghatározásának módszerei, meghatározásuk számítógépes támogatása.}
%----------------------------------------------------------------------------
\subsection{Markov-láncok, születési-kihalási folyamatok.}
A Markov-lánc fogalmához legegyszerűbben a független kísérletek fogalmának általánosításával jutunk el. Tekintsük egymás után végrehajtott  kísérletek sorozatát. Legyen $E_1,E_2,\dots,E_i,\dots$ egy teljes eseményrendszer. Vizsgáljuk az egyes kísérletek eredményét az $E_i$ események bekövetkezése szempontjából. Definiáljuk a $\xi_n\quad(n=0,1,2,\dots)$ valószínűségi változókat úgy, hogy $\xi_n=i$, ha az n-edik kísérletnél az $E_i$ esemény fordul elő.

Független kísérletek esetén érvényes, hogy
$$P(\xi_n=j|\xi_1=i_1,\xi_2=i_2,\dots,\xi_{n-1}=i_{n-i})=P(\xi_n=j)$$
minden n-re, és a szóban forgó valószínűségi változók valamennyi lehetséges értéke. A Markov-lánc fogalmához akkor jutunk, ha a fenti valószínűségeknél feltesszük, hogy a $\xi_n$ változó eloszlása függ az előző $\xi_1,\dots,\xi_{n-1}$ változóktól is.

Ha fennáll minden lehetséges n-re és a változók összes lehetséges értékeire, hogy
$$P(\xi_n=j|\xi_1=i_1,\xi_2=i_2,\dots,\xi_{n-1}=i_{n-i})=P(\xi_n=j|\xi_{n-1}=i_{n-i})$$
úgy azt mondjuk, hogy az egymást követő kísérletek, ill. a $\xi_n$ valószínűségi változók \emph{egyszerű Markov-láncot} alkotnak.

A Markov-láncok fontos speciális esetét képezik a \emph{homogén Markov-láncok}. Ezeknél a  $P(\xi_n=j|\xi_{n-1}=i)$ átmenetvalószínűségek  függetlenek az n-től, azaz
$$P(\xi_n=j|\xi_{n-1}=i)=p_{ij}$$
írható.

Fizikai alkalmazásokat szem előtt tartva a Markov-láncokkal kapcsolatban rendszerint a következő terminológia szokásos: az $E_i$ eseményeket a rendszer \emph{állapotainak} nevezzük. a $\xi_0$ változó $P(\xi_0=i)=P_i(0)$ eloszlást \emph{kezdeti eloszlásnak} és a $P(\xi_n=j|\xi_{n-1}=i)$ feltételes valószínűséget \emph{átmenetvalószínűségnek} nevezzük. Ha pedig $\xi_{n-1}=i$ és $\xi_n=j$ akkor azt mondjuk, hogy a rendszer az n-edik lépésben \emph{átmenetet} tett.

Ha egy Markov-láncnál ismerjük a kezdeti eloszlást és az átmenetvalószínűségeket, úgy ezek segítségével az összes $\xi_n$ változó eloszlása egyértelműen meghatározható.

\subsubsection{Születési-kihalási folyamatok}
A Markov-folyamatok egy igen fontos speciális osztálya születési-halálozási folyamatok néven ismert. A definiáló feltétel: minden állapotból csak "szomszédos" állapotba mehet végbe átmenet ($\pm1$-el változhat). Állapottérnek ekkor a nemnegatív egész számok halmazát választjuk (ami nem megy az általánosság rovására), és $X_k=i$ esetében $X_{k+1}$ vagy $i-1$ vagy $i$ vagy $i+1$ lehet.

Ahhoz, hogy egy $X(t)$ Markov-lánc születési-halálozási folyamat legyen, ki kell elégítenie az alábbi feltételeket:
\begin{enumerate}
	\item $$P(X(t+h)=k+1|X(t)=k)=\lambda_k h+o(h)$$
	\item $$P(X(t+h)=k-1|X(t)=k)=\mu_k h+o(h)$$
	\item $$P(X(t+h)=k|X(t)=k)=1-(\lambda_k+\mu_k)h+o(h)$$
	\item $$P(X(t+h)=m|X(t)=k)=o(h)\quad |m-k|>1$$
\end{enumerate}
ahol $h$ egy tetszőleges kis intervallumot jelent, $o(h)$ pedig olyan mennyiséget jelöl, amely gyorsabban tart 0-hoz, mint $h$, ahogy $h$ nullához tart, vagyis
$$\frac{o(h)}{h}\to 0 ,\quad \text{ha} h\to0$$

Vegyük észre, hogy $\lambda_k,\mu_k$ pozitív mennyiségek függetlenek az időtől. A $\lambda_k$-kat \emph{születési intenzitásnak}, a $\mu_k$-kat pedig \emph{halálozási intenzitásnak} nevezzük.

Jelöljük $P_k(t)$-vel annak a valószínűségét, hogy a folyamat a $t$ időpillanatban a $k$ állapotban van, vagyis
$$P_k(t)=P(X(t)=k)$$
Ezt szokás \emph{abszolút valószínűségnek} is nevezni. Ezen valószínűségek kiszámításához figyelembe kell venni a következőket: A $t+h$ időpillanatban az $X(t)$ $k$ állapotban van akkor és csak akkor, ha az alábbi feltételek teljesülnek:
\begin{enumerate}
	\item $t$ időpillanatban a folyamat a $k$ állapotban van, és a $(t,t+h)$ időintervallumban változás nem következik be
	\item $t$ időpillanatban a folyamat a $k-1$ állapotban volt, és a $k$-ba történt átmenet
	\item $t$ időpillanatban a folyamat a $k+1$ állapotban volt, és a $k$-ba történt átmenet
	\item $(t,t+h)$ alatt 2 vagy több átmenet történt.
\end{enumerate}
Látható, hogy az 1-3 feltételek kölcsönösen kizárják egymást és a 4. eset valószínűsége $o(h)$. Világos, hogy $t$ minden értékére teljes eseményrendszerről van szó, így:
$$\sum_{k=0}^{\infty}P_k(t)=1$$
Az előbbi feltételek teljesülése után mára felírhatjuk a $P_k(t+h)$ valószínűséget:
$$P_k(t+h)=P_k(t)\{1-\lambda_k h-o(h)\}+P_{k-1}(t)\{\lambda_{k-1} h+o(h)\}+P_{k+1}(t)\{\mu_{k+1} h+o(h)\}+o(h),\quad k\ge1$$

\subsection{A legalapvetőbb sorbanállási rendszerek vizsgálata.}
\subsubsection{Rendszerjellemzők}

\paragraph{Kendall-féle jelölés: N/A/B/m/K}~\\
N: igényforrások száma\\
A: beérkezési időközök eloszlása\\
B: kiszolgálási időközök eloszlása\\
m: a kiszolgálók száma\\
K: a várakozási sor kapacitása

\subsubsection{M/M/1}
Az M/M/1 rendszer a legegyszerűbb nemtriviális rendszer. A beérkezési folyamat \textbf{$\lambda$ paraméterű Poisson--folyamat}, vagyis a beérkezési időközök \emph{$\lambda$ paraméterű exponenciális eloszlású} valószínűségi változók.

A kiszolgálási időközök $\mu$ paraméterű exponenciális eloszlású valószínűségi változók. feltesszük továbbá, hogy a beérkezési időközök és a kiszolgálási idők egymástól független valószínűségi változók.

Az ergodikusság szükséges és elégséges feltétele az M/M/1 sor eseten egyszerűen
$$\lambda<\mu,\quad P_0 = 1-\frac{\lambda}{\mu}$$
A kihasználtsági tényező: $\rho=\frac{\lambda}{\mu}$\\
A k.-adik állapot valószínűsége:
$$P_k=(1-\rho)\rho^k \quad k=0,1,2,\dots$$
\paragraph{A rendszer jellemzői}
\begin{description}
	\item[A rendszerben tartózkodó igények átlagos száma] $N=\frac{\rho}{1-\rho}$
	\item[A rendszerben tartózkodó igények átlagos számának szórásnégyzete] $\sigma_N^2=\frac{\rho}{(1-\rho)^2}$
	\item[A várakozó igények átlagos száma (átlagos sorhossz)] $\overline{Q}=\overline{N}-\rho=\frac{\rho^2}{1-\rho}$
	\item[A szerver kihasználtsága] $U_s=1-P_0=\frac{\lambda}{\mu}=\rho$
	\item[A kiszolgáló átlagos foglaltsági periódushossza:] $E_\delta=\frac{1}{\lambda}\cdot\frac{\rho}{1-\rho}=\frac{1}{\lambda}\overline{N}=\frac{1}{\mu-\lambda}$
\end{description}
\subparagraph{Little-formulák}
$$\lambda T=\lambda\frac{1}{\mu(1-\rho)}=\frac{\rho}{1-\rho}=\overline{N}$$
$$\lambda \overline{W}=\lambda\frac{\rho}{\mu(1-\rho)}=\frac{\rho^2}{1-\rho}=\overline{Q}$$

\subsubsection{M/M/1/K}
Rögzített a várakozó igények számának maximuma. Felteszük, hogy a rendszerben legfeljebb K igény tartózkodhat (beleértve a kiszolgálás alatt álló igényt is). Az ezen felül érkező igények nem kerül be a rendszerbe, azonnal távozik, nem kerül kiszolgálásra. Továbbra is \textbf{Poisson--folyamat} szerint érkeznek az igények, azonban csak azok az igények léphetnek be a rendszerbe, amelyek érkezésekor a rendszer állapota kisebb, mint K.

$$\lambda_k=
\begin{cases}
\lambda, & \text{ha } k<K\\
0, & \text{ha } k\geq K 
\end{cases}$$
$$ \mu_k=\mu \quad k=1,2,\dots,K$$
Látszik, hogy ez a rendszer mindig ergodikus, mert állapottere véges. Továbbá:
$$P_k = 
\begin{cases}
P_0\prod_{j=0}^{k-1}\frac{\lambda}{\mu}=P_0(\frac{\lambda}{\mu})^k & \text{ha }k\leq K\\
0 & \text{ha }k>K
\end{cases}$$
$$P_0 = \frac{1-\nicefrac{\lambda}{\mu}}{1-(\nicefrac{\lambda}{\mu})^{K+1}}$$
K=1 esetén a M/M/1/1 rendszer azt jelenti, hogy egyáltalán nincs várakozás.

\subsubsection{M/M/N}
A rendszer n db kiszolgálócsatornával (szerverrel) van ellátva.
Az ergodikusság feltétele: $\frac{\lambda}{n\mu}<1$\\
Ha $k<n$:
$$P_k=P_0(\frac{\lambda}{\mu})^k\frac{1}{k!}$$
Ha viszont $k\geq n$:
$$P_K=P_0(\frac{\lambda}{\mu})^k\frac{1}{n!n^{k-n}}$$

\subsection{A rendszerjellemzők meghatározásának módszerei, meghatározásuk számítógépes támogatása.}

\subsubsection{M/M/n/n -- Elang-féle veszteséges rendszer}
Az $n$ csatornás rendszerbe Poisson--folyamat szerint érkeznek az igények. Ha van üres csatorna (vagy szerver), az igény kiszolgálása exponenciális időtartalmú $\mu$ paraméterrel. Ha minden kiszolgálóegység foglalt, akkor az igény elvész, azaz nincs sorban állás. Ezen probléma a tömegkiszolgálás egyik legrégebbi problémája, mellyel a XX. század elején a telefonközpontok kihasználtságával kapcsolatban foglalkozott A. K. Erlang és C. Palm.
$$\lambda_k=
\begin{cases}
\lambda & \text{ha } k<n\\
0 & \text{ha } k\geq n
\end{cases}$$
$$\mu_k=k\mu\quad k=1,2,\dots,K$$
Azt mondjuk a rendszer k állapotban van, ha k szerver foglalt, azaz ha k igény tartózkodik a rendszerben.

A folyamat stacionárius eloszlása:
$$P_k=
\begin{cases}
P_0(\frac{\lambda}{\mu})^k\frac{1}{k!} & \text{ha } k\leq n\\
0 & \text{ha } k>n
\end{cases}$$
A normalizáló feltétel miatt:
$$P_0=(\sum_{k=0}^{n}(\frac{\lambda}{\mu})^k\frac{1}{k!})^{-1} \implies
P_k=\frac{\frac{\rho^k}{k!}}{\sum_{i=0}^{n}\frac{\rho^i}{i!}}\quad k\leq n$$
A rendszer egyik jellemzője a $P_n=\nicefrac{\frac{\rho^n}{n!}}{\sum_{k=0}^{n}\frac{\rho^k}{k!}}$ valószínűség, melyet először Erlang vezetett be 1917-ben és \emph{Erlang--féle veszteségformula} vagy \emph{Erlang--féle B formula} néven ismert. Általában $B(n,\nicefrac{\lambda}{\mu}$ szimbólummal jelölik. A $P_n$ valószínűség annak a valószínűsége stacionárius esetben, hogy egy újonnan érkező igényt nem fogad a rendszer, azaz az igény elveszik.

Ez a rendszerjellemző meghatározható rekurzív módon, így a köv. formulával számítógép segítségével gyorsan számolható a $B(n,\rho)$:
\begin{align*}
	B(0,\rho)&=1\\
	B(1,\rho)&=\frac{\rho}{1+\rho}\\
	B(n,\rho)&=\frac{\rho B(n-1,\rho}{1+\rho B(n-1,\rho)}
\end{align*}

További számítógépes segítség a rendszerjellemzők meghatározására a szimuláció. A szimulációs folyamat után statisztikai eszközökkel határozzuk meg a rendszerjellemzőket.

