%----------------------------------------------------------------------------
\section{Bevezetés a Cisco eszközök programozásába 2}
{\footnotesize A forgalomirányítási táblázatok felépítése, statikus és dinamikus routing összehasonlítása.}
%----------------------------------------------------------------------------
\subsection{A forgalomirányítási táblázatok felépítése}
A forgalomirányítás a csomagok (IP datagramok) továbbítási irányának meghatározásával kapcsolatos döntések meghozatala. A forgalomirányításhoz szükséges információkat ún. forgalomirányítási táblázatban tartjuk nyilván, melynek segítségével gyorsan meghatározható, hogy egy input interfészen érkező csomagot melyik output interfészen kell továbbítani.\\
\begin{tabular}{|l|l|l|l|l|}
	\hline 
	Célhálózat & Netmask & Kimenő interfész & Köv. Hop & Metrika \\ 
	\hline 
\end{tabular}
\begin{easylist}
	# A router az input interfészen érkező csomagot fogadja.
	# A router a csomag célcímét illeszti a routing táblázat soraira.
	## Ha a célcím több sorra illeszkedik, akkor a leghosszabb prefix sort tekintjük illeszkedőnek.
	# Ha nem létezik illeszkedő sor, akkor a cél elérhetetlen, a csomag nem továbbítható??
	## A csomagot a router eldobja és ICMP hibajelzést küld a feladónak VAGY alapértelmezett átjárón továbbküldi
	# Ha létezik illeszkedő sor, akkor a csomagot az ebben szereplő kimeneti interfészen továbbítjuk (adatkapcsolati rétegbeli beágyazással) a következő, hop-ként megadott szomszédhoz, ill. a célállomáshoz, ha már nincs több hop.
\end{easylist}
~\\
\begin{easylist}
	# A routing tábla sorait prefix hossz szerint csökken, sorrendbe rendezzük. N=1.
	## Ezzel biztosítjuk, hogy több illeszkedő sor esetén a leghosszabb prefixűt fogjuk eredményként kapni.
	# Ha nem létezik a táblázatban az N. sor, akkor nincs illeszkedő sor és vége.
	# A csomag célcíme és az N. sor hálózati maszkja között bitenkénti AND műveletet hajtunk végre.
	# Ha a bitenkénti AND művelet eredménye megegyezik az N. sor célhálózat értékével, akkor a cím az N. sorra illeszkedik és vége.
	# N=N+1, és folytassuk a 2. pontnál.
\end{easylist}

Forgalomirányítási táblázatok feltöltése történhet statikus vagy dinamikus módon. Statikusnál
direkt módon megadunk egy vagy több sort a táblázatba, míg dinamikus esetben routing
protokollok végzik automatikusan, figyelve a hálózatot. Pl.: RIP, OSPF

\subsection{statikus és dinamikus routing összehasonlítása}
\begin{description}[nosep]
	\item[nem adaptív] statikus, nem támaszkodhat döntéseiben a mérésekre, a használandó útvonalakat előre definiálják a routerek indulásakor, bekapcsolásakor. Fix-egyutas esetben egyetlen irányba történik az adattovábbítás, központi forgalomirányító táblát alkalmaznak, vonal szakadása esetén 100\%-os veszteség várható. A fix többutas esetben alternatív utakat kínál fel a csomópontoknak. Ezek közül egy kívülről meghatározott módon (például random) választ egyet.Ekkor egy esetleges vonalszakadás nem okozza minden csomag elveszését.
	\item[adaptív] dinamikus, a forgalomirányítás döntésiben figyelembe veszik a topológiában és a forgalomban történt változásokat. A frissítendő adatokat a (helyi, szomszédos, összes többi) routertől kapja meg.
	\item[Statikus routing] A forgalomirányítási táblázatot a rendszeradminisztrátor tartja karban. Amit egyszer beírunk, az marad a cím. A világ felé egy, vagy csak néhány kapcsolódási pont van.
	\item[Dinamikus routing] Ha több kapcsolódási pont van. Hogyan tartják karban a táblát? A forgalomirányítási táblázat(ok) valamilyen routing protocol segítségével kerülnek karbantartásra.
\end{description}

A dinamikus forgalomirányítás osztályozása:
\begin{description}[nosep]
	\item [Belső forgalomirányíási protokollok] legfőbb alapelv a ,,legjobb útvonal'' meghatározása. (IGP, RIP, OSPF)
	\item[külső forgalomirányítási protokollok] Nem feltétlenül a legjobb útvonal megállapítása a cél, politikai alapú forgalomirányítás. (EGP, BGP)
\end{description}

\subsubsection{A dinamikus routing áttekintése}
Az irányító protokollok a forgalomirányítók között folytatnak adatcserét. Az irányító
protokollok segítségével a forgalomirányítók megoszthatják az általuk ismert hálózatokról
rendelkezésükre álló információkat a többi forgalomirányítóval, illetve közölhetik más
forgalomirányítóktól való távolságukat. A forgalomirányítók a többi forgalomirányítótól
kapott információk alapján saját irányítótáblát építenek fel és tartanak karban.

Példák irányító protokollokra:
\begin{itemize}[nosep]
	\item RIP (Routing Information Protocol)
	\item IGRP (Interior Gateway Routing Protocol)
	\item EIGRP (Enhanced Interior Gateway Routing Protocol)
	\item OSPF (Open Shortest Path First)
\end{itemize}

Az irányított protokollok a felhasználók adatait továbbítják. Az irányított protokollok
elegendő hálózati rétegbeli információt biztosítanak ahhoz, hogy a csomagok a címzési séma
alapján eljussanak a célállomáshoz.

Példák irányított protokollokra:
\begin{itemize}[nosep]
	\item IP (Internet Protocol)
	\item IPX (Internetwork Packet Exchange)
\end{itemize}

\subsubsection{Autonóm rendszerek}
Az autonóm rendszer egyazon felügyelet alá tartozó, közös irányítási stratégiát alkalmazó
hálózatok együttese. A külvilág számára az autonóm rendszer egyetlen entitásnak látszik. Az
autonóm rendszer üzemeltetését egy vagy több operátor is végezheti, ám a külvilág számára
mindig egységes forgalomirányítási képet tükröz.

Az autonóm rendszerek számára az ARIN (American Registry of Internet Numbers) nevű
szervezet, valamilyen szolgáltató vagy egy rendszergazda osztja ki az azonosítókat. Az
autonómrendszer-azonosító egy 16 bites szám. Az irányító protokollok – például a Cisco
IGRP – egy egyedi autonómrendszer-azonosító hozzárendelését követelik meg.

\subsubsection{Az irányító protokollok és az autonóm rendszerek feladata}
Az irányító protokollok feladata az irányítótábla létrehozása és karbantartása. Ez a táblázat a
megismert hálózatokat és a hozzájuk rendelt portokat tartalmazza. A forgalomirányítók az
irányító protokollokat a más forgalomirányítóktól kapott és a saját interfészeik
konfigurációjából felismert információk kezelésére használják, kiegészítve velük a kézzel
konfigurált útvonalakat.

Az irányító protokoll minden rendelkezésre álló útvonalat felmér, a legjobbakat az
irányítótáblába írja, az érvénytelenné váltakat pedig törli. A forgalomirányító az
irányítótáblában lévő információk alapján továbbítja az irányított protokollok csomagjait.

Az irányító algoritmus a dinamikus forgalomirányítás legfontosabb eleme. Amikor a hálózat
topológiája a hálózat növekedése, átkonfigurálása vagy meghibásodása folytán megváltozik, a
hálózatot leíró tudásbázisnak is követnie kell a változást. A tudásbázisnak pontos, konzisztens
képet kell adnia az új topológiáról.

Amikor már az összekapcsolt hálózat valamennyi forgalomirányítója ugyanazt a megváltozott
információbázist használja, akkor azt mondjuk, hogy a hálózati konvergencia megtörtént.
Kívánatos, hogy a hálózat gyorsan konvergáljon, mert így csökken az az idő, amíg a
forgalomirányítók a régi információk alapján helytelen forgalomirányítási döntéseket hoznak.

Az autonóm rendszereknek köszönhetően a világméretű összekapcsolt hálózat kisebb,
könnyebben kezelhető hálózatokra oszlik. Minden autonóm rendszer saját szabálykészlettel és
házirenddel rendelkezik, valamint egyedi autonómrendszer-azonosítója van, amely a világ
összes más autonóm rendszerétől megkülönböztethetővé teszi.

\subsubsection{Az irányító protokollok osztályai}
A legtöbb irányító algoritmus besorolható a következő két alapvető osztály valamelyikébe:
\begin{itemize}
	\item távolságvektor alapú
	\item kapcsolatállapot alapú
\end{itemize}

A távolságvektor alapú eljárás az összekapcsolt hálózatban minden összeköttetéshez egy
irányt (vektort) és egy távolságot határoz meg.

A kapcsolatállapot alapú, más néven legrövidebb utat kereső módszer a teljes összekapcsolt
hálózatról térképet készít.

\subsubsection{A távolságvektor alapú forgalomirányító protokollok szolgáltatásai}
A távolságvektor alapú irányító algoritmusokat használó forgalomirányítók periodikusan
elküldenek egymásnak egy-egy másolatot az irányítótáblájukról. Ezek a rendszeres frissítések
viszik át a topológia változásaira vonatkozó információkat a forgalomirányítók között. A
távolságvektor alapú algoritmusokat Bellman-Ford algoritmusoknak is nevezik.

Minden forgalomirányító a közvetlen szomszédjaitól kap irányítótáblákat. A B
forgalomirányító az A forgalomirányítótól kap adatokat. A B forgalomirányító a kapott
adatokhoz valamilyen távolságadatot ad hozzá, például az ugrások számát, így növeli a
távolságvektor nagyságát. A B forgalomirányító ezt a módosított irányítótáblát adja tovább
szomszédjának, a C forgalomirányítónak. A szomszédos forgalomirányítók között lépésről
lépésre ugyanez a folyamat játszódik le.

Az összegyűlt távolságértékek alapján az algoritmus egy adatbázist tart karban a hálózat
topológiájáról. A távolságvektor alapú algoritmusokkal azonban nem alkotható teljes kép az
összekapcsolt hálózatról, mivel minden forgalomirányító csak a szomszédos
forgalomirányítókat látja.

A távolságvektor alapú forgalomirányítást használó forgalomirányítók először szomszédjaikat
azonosítják. A közvetlenül csatlakoztatott hálózatok felé vezető interfészek távolsága nulla.
A távolságvektor alapú felismerő folyamat előrehaladtával a forgalomirányítók a különféle
célhálózatok felé vezető legjobb útvonalakat a szomszédjaiktól kapott adatok alapján ismerik
fel. Az A forgalomirányító a B forgalomirányítótól kapott adatok alapján ismeri meg a többi
hálózatot. Az irányítótábla minden más bejegyzése akkumulált távolságvektort tartalmaz,
amely az adott hálózat adott irányban mért távolságát tükrözi.

Az irányítótábla frissítései a topológia változásait követően történnek meg. A hálózatfelderítő
folyamathoz hasonlóan a topológiaváltozásra vonatkozó frissítések is lépésenként jutnak el az
egyik forgalomirányítótól a másikig. A távolságvektor alapú irányító algoritmusokat futtató
forgalomirányítók a teljes irányítótáblájukat elküldik valamennyi szomszédjuknak. Az
irányítótáblák többek között az útvonalaknak az adott mérték szerinti teljes költségét és az
egyes hálózatokhoz vezető útvonalak első forgalomirányítójának logikai címét tartalmazzák.

A távolságvektorokat az autópályák kereszteződéseiben látható feliratokhoz hasonlíthatnánk.
Minden csomópontban nyíl mutatja a célállomások irányát, távolságukat pedig szám jelzi. Ha
továbbhaladunk, akkor újabb nyíl mutatja a helyes irányt, de már kisebb távolság lesz kiírva.
Amíg a távolság csökken, addig tudhatjuk, hogy a legjobb úton haladunk.

\subsubsection{A kapcsolatállapot alapú irányító protokollok szolgáltatásai}
A kapcsolatállapot alapú algoritmusokat Dijkstra-algoritmusoknak vagy legrövidebb utat
kereső algoritmusoknak is nevezzük.\\
A kapcsolatállapot alapú irányító algoritmusok a topológiáról szerzett információból összetett
adatbázist készítenek. A távolságvektor alapú algoritmusok a távoli hálózatokról nem
rendelkeznek specifikus információkkal, és a távoli forgalomirányítókat sem ismerik. A kapcsolatállapot alapú algoritmusok ezzel szemben minden adattal rendelkeznek a távoli
forgalomirányítókról és kapcsolataikról.

A kapcsolatállapot alapú forgalomirányítás a következő adatokat, eszközöket használja:
\begin{description}[nosep]
	\item[Kapcsolatállapot-hirdetések (link-state advertisement, LSA)] A kapcsolatállapot-hirdetés egy forgalomirányítási adatokat tartalmazó kisméretű csomag. A forgalomirányítók továbbítják egymás között.
	\item[Topológiai adatbázis] A topológiai adatbázis az LSA-kból kinyert információkat foglalja össze.
	\item[Legrövidebb utat kereső algoritmus] A legrövidebb utat kereső (shortest path first, SPF) algoritmus az adatbázis adatait dolgozza fel. Eredményként az SPF-fát adja.
	\item[Irányítótáblák] Az ismert útvonalak és interfészek listája.
\end{description}

A kapcsolatállapot alapú forgalomirányítás által alkalmazott hálózatfelderítési módszerek
Az LSA-k kölcsönös elküldése a forgalomirányítók között a közvetlenül csatlakozó hálózatok
adataival kezdődik, ugyanis ezekről a forgalomirányítók közvetlen adatokkal rendelkeznek. A
többiekkel párhuzamosan az LSA-k felhasználásával az összes forgalomirányító létrehoz egy
topológiai adatbázist.

Az SPF algoritmus a hálózatok elérhetőségét határozza meg. A forgalomirányító a logikai
topológiát egy fába szervezi, amelynek saját maga a gyökere. A fa a kapcsolatállapot alapú
forgalomirányítást használó összekapcsolt hálózat összes hálózatához minden lehetséges
útvonalat tartalmaz. Ezután az algoritmus az útvonalakat hosszuk szerint sorba állítja úgy,
hogy a legrövidebb útvonal kerüljön előre (innen származik az SPF, „shortest path first"
elnevezés). A forgalomirányító a legjobb útvonalakat és a hozzájuk tartozó interfészeket az
irányítótáblájába írja. Ezenfelül a hálózati topológia elemeiről és azok állapotáról további
adatbázisokat tart karban.

Az a forgalomirányító, amely elsőként szerez tudomást valamilyen kapcsolat állapotának
megváltozásáról, ezt az információt továbbítja, így a többi forgalomirányító megfelelő
frissítést tud végezni. A folyamatba az is beletartozik, hogy általános irányítási információkat
küldenek az összekapcsolt hálózat valamennyi forgalomirányítójának. A konvergencia
megvalósítása érdekében minden forgalomirányító nyilvántartja szomszédjait, nevüket,
interfészeik állapotát, illetve a hozzájuk vezető összeköttetés költségét. Minden
forgalomirányító egy LSA-csomagot készít, amely az új szomszédokkal, az összeköttetések
költségének megváltozásával és az érvénytelenné vált összeköttetésekkel kapcsolatos adatokat
tartalmazza. Ezután elküldi az LSA-csomagot, amelyet az összes forgalomirányító
megkaphat.

Amikor a forgalomirányító LSA-csomagot kap, az adatbázist a legújabb adatokkal frissíti,
ezek alapján újraszámítja az összekapcsolt hálózat térképét, valamint az SPF algoritmussal
meghatározza az egyes hálózatokhoz vezető legrövidebb útvonalakat. Valahányszor egy
beérkező LSA-csomag változást okoz a kapcsolatállapotokat tároló adatbázisban, az SPF
algoritmus újraszámolja a legjobb útvonalakat, és ennek megfelelően módosítja az
irányítótáblát.

A kapcsolatállapot alapú irányítás problémái:
\begin{itemize}[nosep]
	\item Nagy processzorterhelés
	\item Komoly memóriakövetelmények
	\item A sávszélesség erőteljes igénybe vétele
\end{itemize}

A kapcsolatállapot alapú protokollokat futtató forgalomirányítóknál több memóriára és
nagyobb processzorteljesítményre van szükség, mint a távolságvektor alapú protokollt
használóknál. A forgalomirányítóknak elegendő memóriával kell rendelkezniük ahhoz, hogy
tárolják a különféle adatbázisokból származó adatokat, a topológiafát és az irányítótáblát. A
kezdeti kapcsolatállapot-csomagok elárasztással történő szétküldése jelentős sávszélességet
köt le. A kezdeti felderítés folyamán a kapcsolatállapot alapú protokollokat használó
valamennyi forgalomirányító az összes többi forgalomirányítónak LSA-csomagot küld. Ezek
a csomagok elárasztják az összekapcsolt hálózatot, és emiatt átmenetileg csökken az irányított
forgalom által felhasználói adatok továbbítására használható sávszélesség. A kezdeti
elárasztás után a kapcsolatállapot alapú irányító protokollok általában csak minimális
sávszélességet igényelnek, mert a forgalomirányítók a topológiaváltozásokat mutató LSA-
csomagokat ritkán vagy eseményvezérelten küldik.

\subsubsection{Autonóm rendszerek, az IGP és az EGP összehasonlítása}
A belső irányító protokollokat több részből álló, de egyetlen szervezet felügyelete alatt lévő
hálózatokhoz tervezték. A belső irányító protokollok esetében a legfontosabb elvárás, hogy
megtalálják a legjobb útvonalat a hálózaton keresztül. Másként fogalmazva mondhatjuk, hogy
a mérték jellege és használatának módja a belső irányító protokollok legfontosabb jellemzője.

A külső irányító protokollokat különálló, eltérő szervezetek ellenőrzése alatt lévő hálózatok
között használják. Általában internetszolgáltatók között, illetve internetszolgáltató–ügyfél
kapcsolatok felett használják őket. Adott cég például saját és internetszolgáltatójának
forgalomirányítója között dönthet a BGP – egy külső irányító protokoll – használata mellett.
Az IP alapú külső átjáróprotokollok esetében a forgalomirányítás megkezdése előtt a
következő három adathalmazra van szükség:
\begin{itemize}
	\item A szomszédos forgalomirányítók listája, ezekkel folyik az irányítási információk cseréje
	\item A közvetlenül elérhető hálózatok listája, melyeket hirdetni kell
	\item A helyi forgalomirányító autonómrendszer-azonosítója
\end{itemize}

A külső irányító protokolloknak el kell különíteniük az autonóm rendszereket. Ne feledjük, az
autonóm rendszerek más és más rendszergazdák felügyelete alatt állnak. A hálózatoknak
biztosítaniuk kell valamilyen protokoll támogatását, amely felett ezek a különböző rendszerek
kommunikálni tudnak egymással.

Az autonóm rendszerek rendelkeznek egy azonosítóval, amelyet az ARIN (American Registry
of Internet Numbers) szervezet vagy egy szolgáltató oszt ki nekik. Az autonómrendszer-
azonosító egy 16 bites szám. Az irányító protokollok – például a Cisco IGRP és EIGRP – egy
egyedi autonómrendszer-azonosító hozzárendelését követelik meg.